% vim: spell ft=tex
\section{Appraisal}
\subsection{Evaluation of objectives}
\boxref{6.1-objectives}

\begin{enumerate}
    \item The user must be able to:
    \begin{enumerate}
        \item Create and edit appointments
        \item Remove appointments (including both finished ones and ones not yet
            passed)
        \item View appointments by day or in a simpler monthly view
        \item Create and edit to-dos
        \item Tick off to-dos as they are completed
        \item Remove to-dos (both completed and not)
    \end{enumerate}
\end{enumerate}

All of these objectives have been met by my software. The monthly view is
limited as it only allows the user to view the current month.

\begin{quotation}
2. The software must be light on memory and processor time usage, as it may be
left to run passively in the background.
\end{quotation}

My program meets this objective. On my computer, running the program has no
discernible impact on system operation. Leaving it running for a long time (6
hours) does not change the resources it uses.

\begin{quotation}
3. The user should be able to view the contents of the application windows
easily even as small windows, since the software may be kept on the desktop
permanently (as a digital to-do list).
\end{quotation}

My solution meets this objective partially. In general, windows and widgets
behave well even when small (600x300 for appointments tab, 300x200 for to-dos
tab). Some widgets fail to render well when made smaller than 200x200. The
appointment tab especially needs to have a large enough height to show the full
calendar without 'squishing' of elements.  However, these situations would only
happen in cases where the window is made very small.

\begin{quotation}
4. The software should allow printing of any calendar views, and a printout of
the current to-do list.
\end{quotation}

This objective is partially fulfilled. When printing the to-do list, only the
pending (incomplete) to-dos are put on the list, as intended. The two calendar
views, daily and monthly, can both be printed out: however they aren't
formatted specially for printing. To fully meet it, each printout should look
like it was made in a word processor, rather than like a screenshot of the program itself.


\subsection{Client feedback}
\boxref{6.2-feedback}

I talked with Amartya about the software, gave him the manual and explained how
to use the program. From this session I received some feedback from him.

``I found the software very easy to understand and move through . Since I use
my laptop so much it's really handy to be able to add a note quickly and not
worry about losing a post-it note or piece of paper. The program loads up and
shuts down very quickly, and the keyboard shortcuts detailed in the manual are
quite helpful since it means I only need to divert my attention for a few
seconds. Because I'm a developer I appreciate that HTML is rendered as the raw
characters rather than butchered formatting! On appointments, I'm pleased with
how you can store multiple different pieces of information separately so that
it's obvious how best to make an appointment and so I can easily see the
important bits. The printouts for appointments are not ideal but I usually only
print my to-dos for the day, so it doesn't affect me too much.

I don't think there needs to be two buttons each for creating to-dos and
appointments, since it seems redundant: I just use the bigger button at the
bottom of each tab. The search results window gives clear information
about the appointments found, but it would be better if I could print them out
like I can with other windows. Also, an edit button for each appointment would
make amending appointments in the future much faster.''


\subsection{Feedback analysis}
\boxref{6.3-feedback-analysis}

My client said he found it easy to move around the program. Though not a
specific objective, this is important because it means it should be accessible
to people who aren't as experienced with computers.

Amartya stated that the program was quick to start and close. This is a great
outcome, since it's a big advantage over online calendars and some desktop ones.

There were many points made on the user interface and how it could be improved.
I was using C++ and Qt for the first time so given my time constraints I kept
some widgets rather basic. With more time and some learning, I could make sure
the program doesn't 'squish' any of its elements, make printouts which look more
natural, and possibly introduce some configurable elements such as font size.


\subsection{Later extensions}
\boxref{6.4-extensions}

Something possible in real life that my software did not emulate is the sorting
and ordering of to-dos. If a user were able to decide on the order of the to-dos
themselves (without removing and re-adding each individual one), it could make
it easier for them to find with a glance what they're looking for -- e.g. by
moving the 'now' items to the top, and the 'pending' further down. A mock-up
could look similar to this:

\addsmallfigure
    {extension-arrow}
    {An example mock-up of using more buttons to reorder to-dos.}
    {fig:extension-arrow}

As an extension to this idea, being able to group to-dos together and only show
certain groups could be helpful. However, although these ideas make viewing the
highest priority tasks quicker and easier, it could seem like unnecessary
'interface cruft' to users who don't require special features for efficiency.
Already each to-do and appointment has two and three buttons respectively linked
to each, so I would be unsure about adding more buttons.

One feature I wanted to add was notifications for appointments. A user would
select an appointment and specify a date and time for a notification about
it to pop up. I encountered difficulties in finding a simple and clean
cross-platform solution for desktop notifications, however, and decided against
implementing it since my client Amartya didn't request it as an important
feature.

The monthly calendar view was an underdeveloped idea which I think could have
some improvements to make it more useful. For example, there could be some
built-in public holidays and days of importance (such as the beginning and end
of BST). Also, it may be easier to understand if it were in the classic format
where the weekdays are the columns, instead of starting every month at 1 on the
left.
