\section{Design}
\subsection{System overview}

\begin{itemize}
    \item \textbf{DBHelper}: An 'interface'/base class which is inherited by
        specific database helpers (e.g. SQLiteHelper).
    \item \textbf{SQLiteHelper}: Handles connections with a SQLite database and
        inserting/selecting data.
    \item \textbf{DataHandler}: Handles retrieving data and updating the
        database. Uses a class derived from DBHelper to interface with the
        actual database.
\end{itemize}


\subsubsection{DataHandler}

Reasoning for this is that it means I can:

\begin{itemize}
    \item Move between databases without changing unrelated code
    \item Easily implement caching if it turns out to be required
\end{itemize}

Separating the generic 'data movement interface' from the specific database
protocol used means that code which handles data does \textit{not} need to know
about the database being used.

\begin{itemize}
    \item \texttt{insert_appts(vector<Appointment>)}
    \item \texttt{select_appts(string title, striang description)}
\end{itemize}


\subsection{System components}
\subsection{IPSO table}

\todo{Um}

\newcommand{\ipsol}[1]{#1 \\}

\starttable{| l | l |}
  \hline
  \thead{Inputs} & \thead{Processes} \\
  \hline
  \ipsol{An input & A process}
  \ipsol{& Another process \\
  \hline
  \thead{Outputs} & \thead{Storage} \\
  \hline
  An output & A storage \\
  \hline
  Another output \\
  \hline
  Another output & Another storage \\
  \hline
\stoptable


\subsection{Design Data Dictionary}
