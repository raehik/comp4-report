\section{Design}
\subsection{System overview}

\begin{itemize}
    \item \textbf{DBHelper}: An 'interface'/base class which is inherited by
        specific database helpers (e.g. SQLiteHelper).
    \item \textbf{SQLiteHelper}: Handles connections with a SQLite database and
        inserting/selecting data.
    \item \textbf{DataHandler}: Handles retrieving data and updating the
        database. Uses a class derived from DBHelper to interface with the
        actual database.
\end{itemize}


\subsubsection{DataHandler}

Reasoning for this is that it means I can:

\begin{itemize}
    \item Move between databases without changing unrelated code
    \item Easily implement caching if it turns out to be required
\end{itemize}

Separating the generic 'data movement interface' from the specific database
protocol used means that code which handles data does \textit{not} need to know
about the database being used.

\begin{itemize}
    \item \verb+insert_appts(vector<Appointment>)+
    \item \verb+select_appts(string title, striang description)+
\end{itemize}


\subsection{System components}
\subsection{IPSO table}

\todo{Use Excel}


\subsection{Design Data Dictionary}
