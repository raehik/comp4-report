\section{Design}
\subsection{System overview}

I am approaching the design of the system modularly i.e. as much of the system
as possible should be 'pluggable'. Where required, classes and modules should be
required to inherit from a base class or module, or be forced to expose certain
functions, as a sort of API. For example, the class for handling data storage
and retrieval could use any object of a 'database helper' type (an interface in
Java), and it would expect the class to know how to handle different functions,
and what return types are required. In this way, a handler for a different
database could be written with minimal code duplication, which would be useful
if I need to change database systems (e.g. to MySQL).

\subsubsection{Main classes}

\newcommand{\classitem}[1]{\item \textbf{#1}}

The main logic classes follow:

\begin{itemize}
    \classitem{DBHelper}: Interface for classes which interface with a specific
    database system (e.g. SQLite, MySQL).
    \classitem{SQLiteHelper}: DBHelper for manipulating a SQLite database
    (inserting/selecting data, executing commands)and inserting/selecting data.
    \classitem{DataHandler}: Handles retrieving data and updating the database.
    Uses a class derived from DBHelper to interface with the actual database.
    \classitem{PDFExporter}
\end{itemize}

\todo{Shit, there needs to be a bit more logic in this. Probably 5 classes at
least. I know DataHandler and DBHelper children do a *lot* of heavy lifting, but
too much is just GUI code.}

\begin{itemize}
    \classitem{GuiMainWindow}: The 'master' GUI window, the main one that the
        user interacts with.
    \classitem{DataEntrySubmitWindow}: Interface for simple dialog windows
        which take user input and use it somehow (e.g. add to a database or make
        a database search)
    \classitem{CalendarViewWindow}: Interface for a calendar view in a separate
        window. Should have an option to print the view.
    \classitem{TodoWidget}: A to-do widget for the main window.
    \classitem{RecentApptsWidget}: A recent appointments widget for the main
    window.
    \classitem{MonthViewWidget}:
\end{itemize}


\subsubsection{DataHandler}

Reasoning for this is that it means I can:

\begin{itemize}
    \item Move between databases without changing unrelated code
    \item Easily implement caching if it turns out to be required
\end{itemize}

Separating the generic 'data movement interface' from the specific database
protocol used means that code which handles data does \textit{not} need to know
about the database being used.

\begin{itemize}
    \item \verb+insert_appts(vector<Appointment>)+
    \item \verb+select_appts(string title, striang description)+
\end{itemize}


\subsection{System components}
\subsection{IPSO table}

\todo{Use Excel}


\subsection{Design Data Dictionary}
