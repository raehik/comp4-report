% vim: spell
\section{Design}
\subsection{System overview}
\subsubsection{Software choices}

The first decision I have is on programming language, which will determine which
toolkits and packages I can use. Since I'm developing a desktop program using a
Linux-based OS, I decided on \textbf{C++} over Java or C\#, because from
experience I've found developing Java on Linux is troublesome, and C\#/.NET is a
Windows-based framework. C++ gives me good flexibility and was made for
object-oriented projects.

In my proposed system, data must be stored between program sessions. The main
database options available to me are:

\begin{itemize}
    \item A plain text file which is loaded into memory every time data needs to
        be accessed (unreliable \& requires further testing, easier to code).
    \item A SQLite database (local, simple to set up, highly tested \&
          reliable).
    \item A MySQL database (remote, requires Internet access, needs
          adminstrating).
\end{itemize}

I will be using SQLite for a local storage database, since it provides reliable
and fast data storage and searching, plus stores its database as a file on the
host computer so can be used offline. If I was using an online solution, I would
use MySQL instead, which provides many more features for efficient queries.

For the user interface, I have only a couple of viable options:

\begin{itemize}
    \item Qt, a C++ framework which provides a GUI framework strongly
          interlinked with Q* utility classes (e.g. QDate)
    \item GTK+, a C/C++ GUI toolkit, providing very similar graphical elements
          but fewer 'extra' features
\end{itemize}

I decided on Qt because it was written in and for C++, and I wouldn't have to
implement so many of my own utility classes for the graphical parts. It also
makes writing cross-platform software easier.


\subsection{System modularity}
\boxref{2.2-modular}

I am approaching the design of the system modularly, i.e. as much of the system
as possible will be \textit{pluggable}. Where required, classes and modules will
be required to inherit from a base class or module, or need to expose certain
functions, as a sort of API or 'contract'. For example, the class for handling
data storage and retrieval could use any generic \textbf{database helper}
object, and it would assume the class knows how to handle certain function
calls, and what return types it expects. In this way, a handler for a different
database could be written with minimal code duplication, which would be useful
if I need/want to change database systems (e.g. to CSV). This design also has
the useful implication of making exporting data into different formats easier (a
potential feature to implement later), or if I need to add a feature like
caching (it could be done almost transparently, with little change to calling
code).

As much as possible, the data classes will not depend on specific features of my
chosen GUI framework. Because I'm using Qt there will be some overlap to
minimise code duplication and unneeded extra work, but I plan only to use Qt
features for things which can easily be swapped out for more abstract classes
later.


\subsection{Design data dictionary}
\boxref{2.3-ddd}
\boxref{2.4-records}

The database for my system will hold two types of records: \textbf{to-dos} and
\textbf{appointments}. Each record has different fields. For each table, field
types are shown next to the field name in brackets for the first row. Each
following row is an example valid record.


\subsubsection{To-do}

\begin{table}[H]
    \centering
    \begin{tabulary}{\linewidth}{|L|L|L|} \hline
            \textbf{ID (integer)} &
            \textbf{Text (string)} &
            \textbf{Completed? (Boolean)} \\ \hline
        1 & Order The \TeX Book                   & false \R
        2 & Buy some milk                         & true  \R
        3 & Check 'characters' allowed in a to-do & true  \R
    \end{tabulary}
    \caption{Example to-do records.}
    \label{tbl:todo-rec}
\end{table}

To-do IDs increment starting from 1. \texttt{Text} can be any string. The
\texttt{Completed?} field defines whether a to-do is 'finished' or not. A
completed to-do will show up at the bottom of the full to-do list and will
be clearly indicated as finished.

All the fields in a to-do are \textbf{required}.


\subsubsection{Appointment}

\begin{table}[H]
    \centering
    \begin{tabulary}{\linewidth}{|L|L|L|L|L|L|} \hline
            \textbf{ID (int.)} &
            \textbf{Title (str.)} &
            \textbf{Date (int.)} &
            \textbf{Description (str.)} &
            \textbf{Time (int.)} &
            \textbf{Location (str.)} \\ \hline
        1 & Work meeting & 2457431 &
            Amy \& Bob presenting their research in big data &
            58800 & Meeting room \R
    \end{tabulary}
    \caption{Example appointment records.}
    \label{tbl:todo-appt}
\end{table}

ID increments from 1. In an appointment record, the \textbf{ID, title, date and
time} fields are required. The location and description are fields for extra
information, which will make it easier to visually search through appointments
for certain events.


\subsection{IPSO table}
\boxref{2.1-overall}

\addfigure
    {ipso}
    {An IPSO table for the planned software.}
    {fig:ipso}


\subsection{Input validation}
\boxref{2.5-validation}

A SQLite database can store any string of characters -- however, if a
user escapes their input in a certain way (on purpose or not), it could
potentially break the record insert statement, which could cause no appointment
to be added. It is not hard for a malicious user to create a query which would
end the insert statement early, then run another statement to e.g. delete all
records in a database.

A programmer can prevent such attacks (known as \textbf{SQL injection
attacks}) by parameterising user input rather than inserting user input as
substrings into a main query string. As long as this is implemented, the user
can input any string and it would be valid for a new record.

The other input validation I need to do is to ensure that all strings are
displayed correctly. Some of Qt's classes will automatically display strings as
formatted HTML, meaning users can create unexpected effects and potential layout
problems if they insert \texttt{<h1>} or other such tags. It is possible to
disable this so that strings are displayed exactly as they were entered.

To-do and appointment records both have some required fields, so when the user
is entering data for a new record those fields should be checked to make sure
they are not empty.


\subsection{Database design}
\boxref{2.6-database-design}

Qt provides a multitude of 'safe' file paths to use for storing user data in,
which (importantly) are guaranteed to work across multiple operating systems. My
program only uses one file, which is the database file.

The database is simple and holds two types of records: 'appointments' and
'to-dos' in their respective databases. (See the Database Records section for
more.)


\subsection{Sample SQL queries}
\boxref{2.7-sql}

In my program, I will have a layer on abstraction on top of SQL, so GUI windows
will not be making SQL queries. Instead they will call a database handler which
in turn calls its specific database helper, which will translate the function
call and parameters into a query for its database, execute it, then return a
result or a code indicating success or failure. I only plan to implement one
database helper for SQLite, but this setup allows better separation of concerns
between GUI elements and the database.

My SQLite database helper will be making many different parameterised queries.
Appointment and to-do records must be INSERTed, UPDATEd, SELECTed and DELETEd.
The SELECT statements will be especially varied, since I'm using two different
views for appointments. Furthermore, to create tables for a new database file, I
need to run EXEC statements.

An example INSERT statement for an appointment would be:

\begin{minted}[breaklines,frame=lines,xleftmargin=15pt]{text}
INSERT INTO appointments(id,title,date,description,time,location)
values(?,?,?,?,?,?);
\end{minted}

Each parameter (indicated by a single question mark) would be bound to a
variable, which prevents the user from escaping the field in a way to end the
query early or otherwise perform unwanted actions. The readied statement could look like this:

\begin{minted}[breaklines,frame=lines,xleftmargin=15pt]{text}
INSERT INTO appointments(id,title,date,description,time,location)
values(1,"Go to the gym",2457501,"",58800,"Local gym");
\end{minted}

In order to SELECT a range of appointments matching certain criteria, I need to
use a WHERE clause. A query to find all appointments with a date before 3rd
March 2016 (Julian day 2457512) would look like so:

\begin{minted}[breaklines,frame=lines,xleftmargin=15pt]{text}
SELECT id,title,date,description,time,location
FROM appointments WHERE date <= 2457512;
\end{minted}


\subsection{Storage media}
\boxref{2.8-storage}

The program is intended to be installed to a user's hard drive. In today's age,
hard drives are cheap and commonplace while also having lots of space.
Furthermore, C++ programs usually compile down to small executables due to years
of compiler optimisation, so even using the rather hefty Qt framework shouldn't
make the output executable more than 10 MB.

The data my program stores is solely textual and numeric, so space is again not
a worry. As an approximation, using one byte for each character and numeric
digit, one appointment might be 200 bytes. 1000 appointments would not even
amount to 1 MB of space -- and this is without the compression that SQLite
applies! It is safe to assume that storage space will not be an issue.


\subsection{Data transformation algorithms}
\boxref{2.9-data-algos}

Dates and times are tricky to handle. If stored in string format (e.g.
\texttt{2016-01-02 15:30}) it becomes overly complex to compare two records and
find which is before the other. Thus I need to represent arbitrary dates and
times as numbers.


\subsubsection{Time}

This is easy for a time: storing the number of minutes (or seconds) since 00:00
would suffice, and make comparisons easy and also less data would be stored than
if a string was used. An example manual implementation using seconds
since midnight follows:

\begin{minted}[breaklines,frame=lines,xleftmargin=15pt]{text}
time_input = "16:20:24"
hours, minutes, seconds = time_split(time_input)
return seconds + (minutes * 60) + (hours * 60 * 60) // = 58824
\end{minted}


\subsubsection{Date}

For a specific day, a common count used is the Julian day, which is an integer
indicating the number of days since January 1, 4713 BC (the start of the Julian
Period). This can represent every and any day a user would need, and again uses
fewer bytes than a string representation. All the following divisions are
\textbf{floored operations} \cite{wiki-julian}:

\begin{minted}[breaklines,frame=lines,xleftmargin=15pt]{text}
day_input = "2010-11-13"
year, month, day = date_split(day_input)
julian_day = (1461 * (year + 4800 + (month - 14)/12))/4 + (367 * (month - 2 - 12 * ((month - 14)/12)))/12 - (3 * ((year + 4900 + (month - 14)/12)/100))/4 + day - 32075
\end{minted}


\subsection{User interface designs \& rationale}
\boxref{2.11-rationale}
\boxref{2.12-input-design}
\boxref{2.13-output-design}

In general, my program needs to be straightforward and 'honest' i.e. a button
should do whatever it would immediately seem to do based on its name and
placement. This is important because though my main client is a highly-competent
computer user, the program should be accessible for older generations who would
like to move to a digital calendar solution and may not have much computer
experience.

My original design for the main window was a 'monolithic' approach: everything
would be viewable from the main window, without having to open sub-windows or
extra dialogs. I felt this would be the best way to allow the user to find what
they need quickest.

\addfigure
    {ui-design-sheet-1}
    {The first plan I came up with for the main window layout.}
    {figure:ui-main-1}

However after some deliberation and a conversation with my client Amartya, I
decided to change to a simpler tabbed design. Most importantly, it allows the
window to be much smaller yet still show all the information, which fulfills one
of the criteria set in the analysis.

\addfigure
    {ui-design-sheet-2}
    {The amended main window design, using a simpler tabbed approach.}
    {figure:ui-main-2}

For the searching feature, my chief concern was with clarity. So rather than
having multiple options for different search types (e.g. 'exact match', 'match
anywhere'), I will keep it simple and only give four input fields to the user by
merging the title and description fields -- the input string for that field will
be used to search both fields in each appointment record. I wasn't sure whether
to combine the location with them too all into one field. My client mentioned he
preferred having a separate location field, so I'm planning to use that.

\addfigure
    {ui-search-appts}
    {The planned appointment search dialog.}
    {figure:ui-search-appts}

The new appointment dialog is similar in layout to the appointment search
window, but has separate fields for the title and description, and the second
calendar widget is a time selector instead.

After searching, if any appointments were found, a dialog must pop up and show
the matching appointments. One of the larger considerations I took when
designing this window was the potential for there to be tens (or maybe hundreds)
of appointments. Because of this, I made sure that the elements would still look
good and lay themselves out correctly even with many appointments present by
using a scroll bar.

\addfigure
    {ui-search-results}
    {The planned appointment search results dialog.}
    {figure:ui-search-results}

Finally, the user should be able to view the appointments for the current month
in a simplistic layout, like how physical calendars have a page for each month.

\addfigure
    {ui-month-view}
    {The planned monthly calendar view window for appointments.}
    {figure:ui-month-view}


\subsection{Data security and integrity}
\boxref{2.14-data-security}
\boxref{2.15-system-security}

My system is intended to be used on a (potentially shared) single desktop
computer. It has no online or network features. The software uses a database
file dependent on the user's account name, and keeps the file in their private
application data folder (i.e. in the \texttt{/home/user} folder in Linux, and in
the \verb|%APPDATA%| folder in Windows, which changes for different users). This
means that anyone with access to a user's operating system account can access
and modify their calendar.

Because I am not using a central database file but instead a separate one for
each user, my security troubles are lessened. I can use the operating system's
inbuilt security features rather than password locking different calendars.
Therefore users should have strong passwords to prevent other unauthorised users
from accessing their calendar. Furthermore, if the users do not lock their
account well, details kept on a to-do list and calendar are not \textit{likely}
to be highly private, so it is not a major threat.

I am using SQLite in a simple manner, meaning every statement I execute is
immediately applied to the database file. This can be slower than batch
execution of statements, but it means data corruption is highly unlikely, since
I/O operations happen in a very short amount of time. SQLite is respected and
well-known for being an incredibly reliable database.


\subsection{Test strategy}

I will split my testing into two parts: testing for data-handling classes
(testing that errors and invalid input are handled well), and GUI testing
(ensuring that buttons work correctly, errors are displayed for invalid input).
I will put emphasis on testing erroneous or dangerous input for record fields,
since problems such as SQL injection are common in systems using SQL databases.
