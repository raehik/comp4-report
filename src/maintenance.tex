% vim: spell ft=tex
\section{System maintenance}
\boxref{4.1-overview}
\subsection{System overview}

The system is formed of two main parts:

\begin{itemize}
    \item The non-GUI data handling classes
    \item The GUI interface for taking input and producing output
\end{itemize}

An abstract view of a database query would go like this:

\begin{itemize}
    \item Data is collected from user using a GUI class
    \item Relevant function call to DataHandler
    \item Relevant function call to DBHelper (SQLiteHelper)
    \item DBHelper translates abstract query to the database language (or
        a series of steps) and updates the database or performs a search
    \item If the function call was expected to return data, simple records are
        returned to DataHandler
    \item DataHandler changes the records into the format expected by the
        original calling GUI class
    \item Data is returned to the GUI class
\end{itemize}

Though the system is best fit for a database like SQLite, by creating another
class which inherits from DBHelper you could create backends for storing data in
specially formatted text files, MySQL, or even a bespoke binary format.


\subsubsection{Database connections}

My program uses a single connection to the database for all operations. Having
multiple connections in SQLite can introduce concurrency problems and can
reduce the reliability of data. To make it easy for windows to use the same
database connection, every window that uses the database takes a \textbf{pointer
to the DataHandler object}. The constructor declaration for GTodoListWidget
looks like this:

\begin{minted}[breaklines,xleftmargin=15pt]{cpp}
GTodoListWidget(DataHandler *db, QWidget *parent = 0);
\end{minted}

Note the \texttt{DataHandler *db} parameter. This is a pointer to a DataHandler
object. Using this, we can access the object's methods by dereferencing the
pointer:

\begin{minted}[breaklines,xleftmargin=15pt]{cpp}
GTodoListWidget(DataHandler *db, QWidget *parent = 0);
\end{minted}


\subsection{Examples of algorithms}

The main classes which handle data in my program are DataHandler and
SQLiteHelper. DataHandler makes abstract data requests/function calls to
SQLiteHelper and runs a small amount of SQL directly for complex functions. The
appointment search window runs \verb|DataHandler::search_appts()|, which runs a
SQLiteHelper function to generate a SQL query such as follows:

\begin{minted}[breaklines,frame=lines,xleftmargin=15pt]{text}
select id,title,date,description,time,location from appointments where title
like '%%' and description like '%%' and date <= 2457536 and date >= 2457484
and location like '%%'
\end{minted}


\subsection{Code listings \& function/variable lists}

The full code listings for the program can be found as a separate document
included with this report. A full list of public and private variables and
methods can be found in the header file section of the code listings.
