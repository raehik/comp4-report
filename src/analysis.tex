\section{Analysis}

\subsection{Testing}

\makeatletter

\meaning\markboth

\meaning\leftmark

\meaning\@leftmark

\meaning\rightmark

\meaning\@rightmark

\meaning\sectionmark

\thesection

\makeatother


\subsection{Background information}

Amartya is a Sixth Form student at Reading School taking 3 A-levels. Classes
often don't set firm homework or deadlines, as teachers expect students to make
time for studying at home themselves depending on how confident they feel with
the material.


\subsection{Identification of problem}

A degree of organisation can help immensely with A-level studies, especially
when we have to find work to do ourselves. Amartya has told me that he has
issues keeping track of work and to-dos, and mentioned that his current methods
for organising projects and appointments are messy.

My project aims to improve Amartya's organisation by making it easier to keep
track of events, appointments and to-dos at various levels of detail - monthly,
daily, hourly.


\subsection{Current system}

Amartya uses post-it notes for both short-term and long-term things (?) (e.g.
both short reminders and deadlines). It can be hard to remember where certain
post-its are, or accidentally place one post-it over another and forget about
the one underneath. Using only post-its makes it a hard-to-use and unorganised
system.


\subsection{End users}

My client is Amartya. I am planning to develop the software around Amartya's
specifications and requirements, but the end users could include me and other
students at the school. I may also be able to market the software given enough
time.


\subsection{Requirements and limitations}

\begin{itemize}
  \item The software must store data in a database reliably, so data
        corruption does not occur.
  \item The software's menus should be straightforward and easy to move
        through.
  \item The software should have an option to make alerts for certain
        events.
  \item The software must be able to show a summary of events in a year
        calendar view, a month view, and a day view.
  \item The software could have 'family sharing', so that authorised
        users can view another person's calendar.
  \item The software must allow searching for events using keywords.
  \item The software must allow searching for events within a range of
        days.
  \item The software could make a printable copy of the different views.
\end{itemize}


\subsection{Data sources and destinations}
\lipsum
\subsection{Data volumes}
\lipsum
\subsection{Analysis data dictionary}

\starttable{| l | l | l |}
  \R \thead{Data} & \thead{Data type} & \thead{Description} \\
  \R Test & Example & Etc.
\stoptable

\subsection{Data flow diagrams}
\subsubsection{Current system}
\subsubsection{Planned system}
\subsection{Entity-relationship model}
\lipsum
\subsection{Object analysis diagram}
\lipsum
\subsection{Objectives}

Numbered!


\subsection{Potential solutions}

\subsubsection{Physical calendar and printouts}
\subsubsection{Pre-existing digital solution}
\subsubsection{Desktop application}

\begin{itemize}
  \item + The software could be used offline.
  \item + Local operations are be faster than querying data on a remote server.
  \item + I wouldn't have to pay for or administrate a server.
  \item - I would have to compile the software for both Windows and Linux.
\end{itemize}


\subsubsection{Web application}

\begin{itemize}
  \item + Immediately compatible with all operating systems
  \item + Simple mobile support
  \item - May be harder to perform some tasks, e.g. printing, event alerts
\end{itemize}


\subsection{Chosen solution}

+ explain how alternative solutions could have been done

\subsection{Etc.}
