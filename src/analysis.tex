\section{Analysis}

\subsection{Background information}

Joe is a student at Reading School currently taking his A-levels. It is
expected of us to make time for study at home as well as going to
classes during school.


\subsection{Identification of problem}

A degree of organisation can help immensely with A-level studies,
especially when we have to find work to do ourselves. Joe has told me
that he has issues keeping track of work and to-dos, and mentioned that
his current methods for organising projects and appointments are messy.

My project aims to improve Joe's organisation by making it easier to
keep track of events, appointments and to-dos at various levels of
detail - monthly, daily, hourly.


\subsection{Current system}

Joe uses post-it notes for both short-term and long-term things (?)
(e.g. both short reminders and deadlines). It can be hard to remember
where certain post-its are, or accidentally place one post-it over
another and forget about the one underneath. Using only post-its makes
it a hard-to-use and unorganised system.


\subsection{End users}

My client is Joe. I am planning to develop the software around Joe's
specifications and requirements, but the end users could include me and
other students at the school. I may also be able to market the software
given enough time.


\subsection{Requirements and limitations}

\begin{itemize}
  \item The software must store data in a database reliably, so
        data corruption does not occur.
  \item The software's menus should be straightforward and easy to move
        through.
  \item The software should have an option to make alerts for certain
        events.
  \item The software must be able to show a summary of events in a
        year calendar view, a month view, and a day view.
  \item The software could have 'family sharing', so that authorised
        users can view another person's calendar.
  \item The software must allow searching for events using keywords.
  \item The software must allow searching for events within a range of
        days.
  \item The software could make a printable copy of the different
        views.
\end{itemize}


\lipsum
\subsection{Data sources and destinations}
\lipsum
\subsection{Data volumes}
\lipsum
\subsection{Analysis data dictionary}

\starttable{| l | l | l |}
    \R \thead{Data} & \thead{Data type} & \thead{Description} \\
    \R Test & Example & Etc.
\stoptable

\subsection{Data flow diagrams}
\subsubsection{Current system}
\lipsum
\subsubsection{Planned system}
\lipsum
\subsection{Entity-relationship model}
\lipsum
\subsection{Object analysis diagram}
\lipsum
\subsection{Objectives}

Numbered!

\subsection{Potential solutions}
\lipsum
\subsection{Chosen solution}

Explain how alternative solutions could have been done.

\subsection{Etc.}
\lipsum
