% vim: spell
\section{Analysis}

\boxref{1.1-bg}
\subsection{Problem identification}

\note{I need this b/c it gives details about my client}
Amartya is a Sixth Form student at Reading School taking 4 A-levels plus an EPQ.
Year 13 is an especially important year for students aiming at university:
pupils' results greatly influence their options following compulsory education.
Classes often don't set firm homework or deadlines, as teachers expect students
to make suitable time for studying at home themselves depending on how confident
they feel with the material.

The support given by the school on organising or prioritising is very limited,
except about revision techniques during examination season. A degree of
organisation can help immensely with A-level studies, especially when we have to
find work to do ourselves. Calendars and other such tools can help with
day-to-day life too. Amartya has told me that he has trouble keeping track of
work and to-dos, and explained that his current methods for organising projects
and appointments are messy.

My project aims to improve Amartya's organisation skills by \textbf{making it
easier} to keep track of events, appointments and to-dos at various levels of
detail including monthly, daily, and hourly, and by reminding the user about
chosen appointments or to-dos.


\boxref{1.2-cur}
\subsection{Current system}

Amartya uses post-it notes for both short-term tasks (such as remembering to
feed the rabbits before night) and long-term things (e.g. deadlines or project
milestones). This simple system works well for short to-do lists or people with
messy desks, since the notes won't get lost under paper. I identified a number
of problems, however:

\begin{itemize}
  \item There is always the concern that a post-it could get buried or fall
        from the wall, meaning a task could be completely forgotten about.
        Also, post-it notes don't provide much space to write on, so it isn't
        always possible to provide extra detail on a longer task; or you have to
        use multiple notes, which requires that they all stay together.
  \item Post-it notes lose adhesiveness over time, so the system would be a pain
        to maintain. This means it \textit{discourages long-term notes} because
        using notes for more than 2-3 weeks makes it more time-consuming to
        re-affix the old ones.
  \item It can be hard to remember where certain post-its are, or you could
        accidentally place one post-it over another and forget about the one
        underneath.
  \item The system doesn't show you what you need to do now or soon -- you have
        to search through the post-it notes and decide for yourself, rather
        than having coming-up events prioritised. This could also mean you have
        to 'remember' which projects you need to work on, which is what an
        organising system should do \textit{for} you.
\end{itemize}

Overall, using only post-its makes a hard-to-use and rudimentary system. My new
system plans to replace the post-it system by with a \textbf{more complete
organiser}: providing advanced features such as calendar functions, to-do
creation, and timeline views with physical printouts, as well as fixing the
problem of priority.


\boxref{1.3-usr}
\subsection{Potential users}

My client for this project is Amartya, who is also my end user. The software
could potentially be used by his family members and other students at my school.
Given more time to polish the software, I may also be able to sell it as a
personal organisation package.

Users will require basic IT competency i.e. keyboard \& mouse to use the
software normally. Some people might want to back up their data, for which they
would need to copy the database file (stored in a user's personal application
data folder) to another location periodically. This folder can move depending
on the host computer. I will add some instructions for this in the user manual
but otherwise assume they have the understanding required.

ICT is taught up to KS3 level in my school, so I can safely assume all students
would be comfortable. Most adults nowadays have basic IT skills, but to cater
for those who are less experienced with computers I will make the manual very
clear and simple to follow.


\boxref{1.4-reqlim}
\subsection{Project requirements}

\begin{itemize}
  \item The software must store data reliably and ensure that data corruption
        does not occur.
  \item The user must be able to create and edit \textbf{appointments}: events
        which associate a sentence or note with a date and time, and extra
        description if required.
  \item The user must also be able to create \textbf{to-dos}, simple notes like
        you would write on a bulleted list.
  \item The software must be able to show a summary of appointments.
\end{itemize}

\todo{Still to-do}
\begin{itemize}
  \item The user must be able to create desktop alerts for appointments, which
        would trigger a notification for an appointment at user-defined time.
  \item The software must allow searching using keywords.
\end{itemize}

\subsubsection{Later additions}
\begin{itemize}
  \item The software could allow users to keep multiple calendars and switch
        between them.
  \item The software could synchronise with other popular calendar services,
        especially Google Calendar.
  \item The software could have 'family sharing', so that authorised users can
        view another person's calendar.
  \item The user could make printable copies of some windows and widgets
        (e.g. print out a display of the months appointments).
\end{itemize}

\subsection{Project limitations}

I will not include synchronisation with other calendar services such as Google
Calendar or CalDAV. My software has an emphasis on offline usage (e.g. for a
laptop) and simplicity, and CalDAV software is not yet widespread, most commonly
used by hobbyists on private servers. Also to match the API for Google Calendar
I would end up duplicating some of its features. I will keep two-way
synchronisation as potential feature for later.


\boxref{1.5-datasrc}
\subsection{Data sources and destinations}

In the current system the user who is organising themselves makes post-it notes.
There is also the option for someone else to write a note and leave it on their
desk. The new system will stay the same in that the only source for to-dos and
appointments is the user owning the calendar (or other people with access to the
user account owning the calendar).\footnote{If a 'family sharing' feature is
implemented, specified users would be able to view or edit another user's
calendar.} With calendar synchronisation, events could be added indirectly,
since any event on the remote calendar would be added to the local calendar on
my software. In general though, it can be assumed calendars are personal and
kept on a user basis.

The entered to-dos or appointments will be stored in a database (in separate
tables) and will be able to be viewed on the calendar views. Notifications be
stored in another database, and store a notification time and a pointer to the
appointment it is notifying about.


\boxref{1.6-datavol}
\subsection{Data volumes}

Only textual data will be stored: times, descriptions etc. are all
non-binary. Data for a single calendar such as appointments and to-dos will be
stored in a single database.

The system can have any number of calendars available. Only one calendar can be
loaded at any time and each calendar is stored in a separate database. I am
making this system expecting that it will be used by a family -- therefore at
most there will be 10 users using the system concurrently, all in different
processes/instances.

Users may access the system up to a few times every hour. The software may be
left running in the background while a computer is on for notifications. Thus
the software should not be resource-intensive while the user is not using it.


\boxref{1.7-add}
\subsection{Analysis data dictionary}

\newcommand{\dictline}[1]{#1 \\ \hline}
\begin{table}[H]
    \centering
    \begin{tabularx}{\linewidth}{|p{0.2\linewidth}|p{0.16\linewidth}|X|} \hline
        \textbf{Data} & \textbf{Data type} &
        \textbf{Description} \R
        \dictline{Appointment & ApptRecord
          & A data structure holding a mandatory ID, title string and date \&
          time, and an optional longer descriptive string, string specifying
          location.}
        \dictline{To-do & TodoRecord
          & A simple data structure containing a mandatory ID, a title string
          and a Boolean determining whether it has been 'completed'.}
        \dictline{Notification & NotifyRecord
          & A data structure holding a mandatory ID, an ID of an appointment
          (ApptRecord), and a date and time to trigger a desktop notification
          at. (The text shown will be the appointment's title string and its
          time.)}
    \end{tabularx}
    \caption{The analysis data dictionary.}
    \label{tbl:add}
\end{table}

\begin{itemize}
    \item \textbf{ApptRecord:}
    \texttt{1 | "Meeting at lunch" | "Another meeting about architecture plans"
    | 2016-04-20 | 12:00 | "Room 101"}
    \item \textbf{TodoRecord:}
    \texttt{7 | "Buy groceries" | False}
    \item \textbf{NotifyRecord:}
    \texttt{2 | 1 | 2016-04-20 | 11:45}
\end{itemize}

\boxref{1.8-dfd}
\subsection{Data flow diagrams}
\subsubsection{Current system}

\newcommand{\addfigure}[3]{
    \begin{figure}[H]
        \centering
        \includegraphics[width=\textwidth]{#1}
        \caption{#2}
        \label{#3}
    \end{figure}
}

\addfigure
    {dfd-orig}
    {DFD of the current post-it note system}
    {fig:dfd-orig}

\subsubsection{Planned system}

\addfigure
    {dfd-new}
    {DFD of the proposed digital system}
    {fig:dfd-new}


\boxref{1.9-er}
\subsection{Entity-relationship model}

\addfigure
    {er-model-new}
    {Entity-relationship model for the proposed system}
    {fig:er-model-new}


\boxref{1.11-objs}
\subsection{Objectives}

\todo{Numbered! Maybe?}

e.g. The user must be able to:

\begin{itemize}
  \item Create new appointments
  \item Edit pre-existing appointments
  \item Remove appointments (including both finished ones and ones not yet passed)
  \item View appointments in a number of ways:
  \begin{enumerate}
      \item View appointments for a chosen day by hour
      \item View appointments for a chosen week (past or future)
      \item View appointments for a chosen month
      \item View appointments for a chosen year
  \end{enumerate}
  \item Print any of the calendar views described above
\end{itemize}

TODO:

All operations should be clear and easy to access (e.g. add to-do,
monthly view).

\begin{itemize}
  \item The software must behave well as a small window, since it's a useful
        application to keep on the desktop permanently (as a digital to-do
        list).
  \item The software's menus should be straightforward and easy to move through.
\end{itemize}


\boxref{1.12-sols}
\subsection{Potential solutions}
\subsubsection{Pre-existing digital solution}

There are a number of existing digital calendars available for varying
platforms. A stand-out contender is \textbf{Google Calendar}, a web-based
calendar which syncs up with Android phones. However, you cannot access your
calendar if you're offline. Also, Google Calendar uses all the details it knows
about you to help you form your appointments and reminders: this is a useful
feature, but some find it invasive to their privacy.


\subsubsection{Excel spreadsheet using macros}

Excel spreadsheets are a simple way to record data in a structured way so that
it can easily be understood. As well, macros give spreadsheets much more
flexibility, since complex functions can be created to speed up certain
operations. However, using this method, the main spreadsheet might have to be
updated continually to work with newer versions of Excel. It could be difficult
to update a calendar to use a new version of the main spreadsheet. The user
would be limited to using Excel for their calendar, and Excel is only available
for Windows and Mac (not Linux-based OSes).


\subsubsection{Desktop application}

The main point for a desktop app is that it can be used offline. Data operations
(e.g. searching for appointments, creating a to-do) are faster than querying for
the same data on a remote server for all but the slowest computers. Since data
would be stored locally, I wouldn't have to pay for or administrate a database
or server. I would need to compile the software for Windows and Linux, but with
some planning it should be simple.


\subsubsection{Web application}

Alternatively I could write my own web-based calendar. This way, it would be
compatible with all major OSes, and mobile support is easy to implement.
Creating complex graphical interfaces is simple in JavaScript compared to
writing in a lower-level language such as C++. However, my application would be
competing with many other web calendars services. Some tasks such as printing
and event alerts may pose more of a challenge to perform in a web browser. I
would need to pay for a server to run my application. Also, I haven't done web
development before, so I would need to spend time learning how to use various
tools in the web development stack.


% table TODO {{{
\newcommand{\solreq}[1]{\textbf{#1}&}
\newcommand{\solreqlast}[1]{\textbf{#1} \\}
\newcommand{\solname}[1]{\textbf{#1}&}
\newcommand{\solY}{\multicolumn{1}{c|}{Yes}}
\newcommand{\solN}{\multicolumn{1}{c|}{No}}

\begin{tabulary}{\textwidth}{|L|L|L|L|} \hline
  \textbf{Possible solutions} &
    % requirements
    \solreq{Interview}
    \solreq{Amartya}
    \solreqlast{and put requirements here}
  \hline
  \solname{Solution 1}
  \solY & \solN & \solY \R
\end{tabulary}
% }}}


\boxref{1.13-chosensol}
\subsection{Chosen solution}

I decided on a desktop 

I chose not to allow swapping between multiple calendars for a single user,
since it adds unnecessary complexity for the user to understand, and if the user
requires specialised and more complex features, there are better services
available -- my program's chief aims are to be quick and easy to understand and
use.

\todo{+ explain how alternative solutions could have been done}
My program will be fully offline and won't gather information about users.

\todo{evidence of use of analysis... 1.14}
