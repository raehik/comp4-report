\section{Analysis}

\subsection{Identification of problem}

Amartya is a Sixth Form student at Reading School taking 4 A-levels. Year 13 is
an especially important year for students aiming at university: pupils' results
greatly influence their options following compulsory education. Classes often
don't set firm homework or deadlines, as teachers expect students to make
suitable time for studying at home themselves depending on how confident they
feel with the material.

The support given by the school on organising or prioritising is very limited,
except about revision techniques during examination season. A degree of
organisation can help immensely with A-level studies, especially when we have to
find work to do ourselves. Amartya has told me that he has trouble keeping track
of work and to-dos, and explained that his current methods for organising
projects and appointments are messy.

My project aims to improve Amartya's organisation skills by \textbf{making it
easier} to keep track of events, appointments and to-dos at various levels of
detail including monthly, daily, and hourly, and by reminding the user about
chosen events.


\subsection{Current system}

Amartya uses post-it notes for both short-term tasks (such as remembering to
feed the rabbits before night) and long-term things (e.g. deadlines or project
milestones). There is always the concern that a post-it could get buried or fall
from the wall, meaning a task could be completely forgotten about. Also, post-it
notes don't provide much space to write on, so it isn't always possible to
provide extra detail on a longer task; or you have to use multiple notes, which
requires that they all stay together.

Post-it notes lose adhesiveness over time, so the system would be a pain to
maintain. This means it \textit{discourages long-term notes} because using notes
for more than 2-3 weeks makes it more time-consuming to affix the old ones.

It can be hard to remember where certain post-its are, or accidentally place one
post-it over another and forget about the one underneath. Overall, using only
post-its makes it a hard-to-use and unorganised system.


\subsection{End users}

My client is Amartya. I am planning to develop the software around Amartya's
specifications and requirements, but the end users could include me and other
students at the school. I may also be able to market the software given enough
time.


\subsection{Requirements and limitations}

\begin{itemize}
  \item The software must store data in a database reliably, so data
        corruption does not occur.
  \item The software's menus should be straightforward and easy to move
        through.
  \item The software should have an option to make alerts for certain
        events.
  \item The software must be able to show a summary of events in a year
        calendar view, a month view, and a day view.
  \item The software could have 'family sharing', so that authorised
        users can view another person's calendar.
  \item The software must allow searching for events using keywords.
  \item The software must allow searching for events within a range of
        days.
  \item The software could make a printable copy of the different views.
\end{itemize}


\subsection{Data sources and destinations}
\lipsum
\subsection{Data volumes}
\lipsum
\subsection{Analysis data dictionary}

\starttable{| l | l | l |}
  \R \thead{Data} & \thead{Data type} & \thead{Description} \\
  \R Test & Example & Etc.
\stoptable

\subsection{Data flow diagrams}
\subsubsection{Current system}
\subsubsection{Planned system}
\subsection{Entity-relationship model}
\lipsum
\subsection{Object analysis diagram}
\lipsum
\subsection{Objectives}

Numbered!


\subsection{Potential solutions}

\subsubsection{Physical calendar and printouts}

Amartya could print or buy calendars and use them to track appointments and
events. This is a common approach to organisation. He could also make weekly
printouts to use more detail or add more events.

Physical calendars are simple, intuitive and powerful. They cannot be used as a
'to-do' tool as effectively as post-it notes, however, since there isn't always
much space for each day. It could lead to a messy desk if the person kept
previous monthly/weekly sheets on hand, and certain events might have to be
transferred between weeks manually if they are ongoing.


\subsubsection{Pre-existing digital solution}

There are a number of existing digital calendars available for varying
platforms. A stand-out contender is Google Calendar


\subsubsection{Desktop application}

\begin{itemize}
  \item + The software could be used offline.
  \item + Data operations are almost always faster than querying data on a
          remote server.
  \item + I wouldn't have to pay for or administrate a server.
  \item - I would have to compile the software for both Windows and Linux.
\end{itemize}


\subsubsection{Web application}

\begin{itemize}
  \item + It would be easy to make it compatible with all major operating
          systems.
  \item + It would be simple to implement mobile support.
  \item - Some tasks may be harder to perform -- e.g. printing, event alerts
  \item - I would be competing with many other web-based calendars.
\end{itemize}


\subsection{Chosen solution}

+ explain how alternative solutions could have been done

\subsection{Etc.}
