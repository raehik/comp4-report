\section{Analysis}

\subsection{Background \& identification of problem}

\note{I need this b/c it gives details about my client}
Amartya is a Sixth Form student at Reading School taking 4 A-levels plus an EPQ.
Year 13 is an especially important year for students aiming at university:
pupils' results greatly influence their options following compulsory
education. Classes often don't set firm homework or deadlines, as teachers
expect students to make suitable time for studying at home themselves depending
on how confident they feel with the material.

The support given by the school on organising or prioritising is very limited,
except about revision techniques during examination season. A degree of
organisation can help immensely with A-level studies, especially when we have to
find work to do ourselves. Amartya has told me that he has trouble keeping track
of work and to-dos, and explained that his current methods for organising
projects and appointments are messy.

My project aims to improve Amartya's organisation skills by \textbf{making it
easier} to keep track of events, appointments and to-dos at various levels of
detail including monthly, daily, and hourly, and by reminding the user about
chosen events.


\subsection{Current system}

Amartya uses post-it notes for both short-term tasks (such as remembering to
feed the rabbits before night) and long-term things (e.g. deadlines or project
milestones). This simple system works well for short to-do lists or people with
messy desks, since the notes won't get lost under paper. I identified a number
of problems, however:

\begin{itemize}

  \item There is always the concern that a post-it could get buried or fall
        from the wall, meaning a task could be completely forgotten about.
        Also, post-it notes don't provide much space to write on, so it isn't
        always possible to provide extra detail on a longer task; or you have to
        use multiple notes, which requires that they all stay together.
  \item Post-it notes lose adhesiveness over time, so the system would be a pain
        to maintain. This means it \textit{discourages long-term notes} because
        using notes for more than 2-3 weeks makes it more time-consuming to
        re-affix the old ones.
  \item It can be hard to remember where certain post-its are, or you could
        accidentally place one post-it over another and forget about the one
        underneath.

\end{itemize}

Overall, using only post-its forms a hard-to-use and rudimentary system. My new
system plans to replace the post-it system by with a more complete organiser:
providing advanced features such as calendar functions, to-do creation, and
timeline views with physical printouts.


\subsection{End users}

My client is Amartya. I am planning to develop the software around Amartya's
specifications and requirements, but the end users could include me and other
students at the school. I may also be able to market the software given enough
time.


\subsection{Requirements and limitations}

\begin{itemize}
  \item The software must store data in a database reliably, so data
        corruption does not occur.
  \item The software must allow searching for events using keywords.
  \item The software must allow searching for events within a range of
        days.
  \item The software must be able to show a summary of events in a year
        calendar view, a month view, and a day view.
  \item The software must have an option to create desktop alerts for selected
        events.
  \item The software's menus should be straightforward and easy to move
        through.
  \item The software could have 'family sharing', so that authorised
        users can view another person's calendar.
  \item The software could make a printable copy of the different views.
\end{itemize}


\subsection{Data sources and destinations}

In the current system the user who is organising themselves makes post-it notes.
There is also the option for someone else to write a note and leave it on their
desk. The new system will stay the same in that the only source is the user
owning the calendar (or other people with access to the computer with the
calendar on).

The entered to-dos or appointments will be stored in a database and will be able
to be viewed on the calendar views. Any events with notifications enabled will
store extra information.


\subsection{Data volumes}

Only textual data will be stored: times, descriptions etc. are all
non-binary. Data for a single calendar such as appointments and to-dos will be
stored in a single database.

Each calendar is intended to be accessed and edited by a single user. If a
'family sharing' feature is implemented, other users would be able to view
another user's calendar (but not edit it).

The system can have any number of calendars available. Only one calendar can be
loaded at any time and each calendar is stored in a separate database. I am
making this system expecting that it will be used by a family -- therefore at
most there will be 10 users using the system concurrently.

Users may access the system up to a few times every hour. The software may be
left running in the background while a computer is on for notifications. Thus
the software should not take resources while the user is not doing operations.


\subsection{Analysis data dictionary}

\starttable{|l|l|l|}
  \hline
  \thead{Data} & \thead{Data type} & \thead{Description} \\
  \hline
  Event & Class & A timed or untimed event/appointment/to-do. \\
  Event title & Example & Etc. \\
  Event location & Example & Etc. \\
  Event description & Example & Etc. \\
  Notification & Class & A class to hold a notification. \\
  Notification time & Date/time & Date/time to show the notification \\
  Notification description & String & Text to show on-screen when notification is active. \\
\stoptable

\subsection{Data flow diagrams}
\subsubsection{Current system}

TODO? Should be easy


\subsubsection{Planned system}
\subsection{Entity-relationship model}

\begin{tikzpicture}[node distance=7em]
\node[entity] (person) {Person};
\node[attribute] (pid) [left of=person] {\key{ID}} edge (person);
\node[attribute] (name) [above left of=person] {Name} edge (person);
\node[multi attribute] (phone) [above of=person] {Phone} edge (person);
\node[attribute] (address) [above right of=person] {Address} edge (person);
\node[attribute] (street) [above right of=address] {Street} edge (address);
\node[attribute] (city) [right of=address] {City} edge (address);
\node[derived attribute] (age) [right of=person] {Age} edge (person);
\node[relationship] (uses) [below of=person] {Uses} edge (person);
\node[entity] (tool) [below of=uses] {Tool} edge[total] (uses);
\node[attribute] (tid) [left of=tool] {\key{ID}} edge (tool);
\node[attribute] (tname) [right of =tool] {Name} edge (tool);
\end{tikzpicture}


\subsection{Object analysis diagram}

TODO


\subsection{Objectives}

\todo{Numbered! Maybe?}

e.g. The user must be able to:

\begin{itemize}
  \item Create new events
  \item Edit pre-existing events
  \item Remove events (including both finished ones and ones not yet passed)
\end{itemize}


\subsection{Potential solutions}

\newcommand{\solreq}[1]{\thead{#1}&}
\newcommand{\solreqlast}[1]{\thead{#1} \\}
\newcommand{\solname}[1]{\thead{#1}&}
\newcommand{\solY}{\multicolumn{1}{c|}{Yes}}
\newcommand{\solN}{\multicolumn{1}{c|}{No}}

\note{I am NOT ALLOWED physical systems here!!! They have to be computer-based
solutions.}

\starttable{*{4}{|p{2cm}}|}
  \hl
  \thead{Possible solutions} &
    % requirements
    \solreq{Interview}
    \solreq{Amartya}
    \solreqlast{and put requirements here}
  \hl
  \solname{Solution 1}
  \solY & \solN & \solY \\
\stoptable


\subsubsection{Pre-existing digital solution}

There are a number of existing digital calendars available for varying
platforms. A stand-out contender is Google Calendar


\subsubsection{Excel spreadsheet}

Excel spreadsheets are a simple way to record data in a structured way so that
it can easily be understood. As well, macros give spreadsheets much more
flexibility, since complex functions can be created to speed up certain
operations.

TODO


\subsubsection{Desktop application}

\begin{itemize}
  \item + The software could be used offline.
  \item + Data operations are almost always faster than querying data on a
          remote server.
  \item + I wouldn't have to pay for or administrate a server.
  \item - I would have to compile the software for both Windows and Linux.
\end{itemize}

\todo{More pluses!}


\subsubsection{Web application}

\begin{itemize}
  \item + It would be easy to make it compatible with all major operating
          systems.
  \item + It would be simple to implement mobile support.
  \item + Graphical interfaces are simple to create.
  \item - Some tasks may be harder to perform -- e.g. printing, event alerts
  \item - I would be competing with many other web-based calendars.
  \item - I would need to spend time learning how to use the various tools in
          the web development stack.
\end{itemize}


\subsection{Chosen solution}

\todo{+ explain how alternative solutions could have been done}

\todo{End of analysis? Check}
