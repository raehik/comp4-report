% vim: spell ft=tex
\section{Testing}
\subsection{Testing plan}
\boxref{3.1-design}

I have some simple unit tests for the SQLiteHelper class to ensure that it
behaves correctly even when given potentially erroneous data. I will tabulate
the results of the unit tests and provide a screenshot of the output logs from
the test run.

For the GUI part of the software, I will focus on testing and validating user
input.


\subsection{Tests}
\subsubsection{Database}

These tests have been adapted from partial unit tests which can be found in
\texttt{testsql.cpp} and \texttt{testgui.cpp}.

\begin{table}[H]
\centering
{ \small
\begin{tabularx}{\linewidth}{|p{0.06\linewidth}|p{0.14\linewidth}|p{0.1\linewidth}|p{0.12\linewidth}|p{0.2\linewidth}|X|} \hline
        \textbf{Test ID} &
        \textbf{Description} &
        \textbf{Test type} &
        \textbf{Data} &
        \textbf{Expected result} &
        \textbf{Actual result} \\ \hline

    SQL00
      & Open new database file
      & Other
      & test-sql.db
      & successful
      & \R

    SQL01
      & Open existing database file
      & Other
      & test-sql.db
      & successful
      & \R

    SQL02
      & Check if a non-existent table exists
      & Other
      & fake\_table
      & no such table
      & \R

    SQL03
      & Pass empty string to \texttt{exec\_sql()}
      & Other
      & ``'' (empty string)
      & take no action
      & \R

    SQL10
      & Select from non-existent table
      & Other
      & fake\_table
      & no such table
      & \R

    SQL11
      & Select from existing table with incorrect columns
      & Other
      & fake\_col1
      & no such column
      & \R

    SQL13
      & Select all from empty table
      & Other
      & fake\_col1
      & no such column
      & \R

\end{tabularx}
}
\caption{Database unit \& other tests}
\label{tbl:test-db}
\end{table}

\addfigure
    {test/sql-test-log}
    {The output log after running the SQL unit tests.}
    {figure:sql-test-log}
